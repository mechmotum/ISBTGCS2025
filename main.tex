\documentclass[11pt,twocolumn]{article}

\usepackage{amsmath}
\usepackage{amssymb}
\usepackage{booktabs}
\usepackage{fancyhdr}
\usepackage[includehead,margin=0.5in]{geometry}
\usepackage{graphicx}
\usepackage{helvet}
\usepackage{parskip}  % disable paragraph indent
\usepackage{sectsty}  % change section header formatting

% helvet font settings
\renewcommand{\familydefault}{\sfdefault}

% make figure caption use "Fig."
\makeatletter
\renewcommand{\fnum@figure}{Fig \thefigure}
\makeatother

% make section headers 11 pt font
\sectionfont{\fontsize{11}{15}\selectfont}

% change bibliography heading
\renewcommand\refname{REFERENCES}

\title{\normalsize\textbf{
  FAST SYMBOLIC METHODS FOR MUSCLE-DRIVEN OPTIMAL CONTROL
}}
\author{
Jason K. Moore\textsuperscript{1} and
Samuel G. Brockie\textsuperscript{2} and
Timótheüs J. Stienstra\textsuperscript{3} and
Antonie van den Bogert\textsuperscript{4} \\
\\
\small
\textsuperscript{1}Department of BioMechanical Engineering, Delft University of Technology, The Netherlands\\
\small
\textsuperscript{2}Department of Engineering, University of Cambridge, United Kingdom\\
\small
\textsuperscript{3}Aerospace Engineering, Delft University of Technology, The Netherlands\\
\small
\textsuperscript{4}Mechanical Engineering, Cleveland State University, USA\\
\small
Email: j.k.moore@tudelft.nl}
\date{}

\renewcommand{\thispagestyle}[1]{} % do nothing
\renewcommand{\headrulewidth}{0pt}

\begin{document}
\pagestyle{fancy}
\lhead{}
\rhead{\textbf{XX International Symposium on Computer Simulation in Biomechanics\\
July 23rd – 25th 2025, Uppsala}}
\fancyfoot{}
\maketitle
\section*{INTRODUCTION}
\vspace{-1em}
%
Direct collocation is now widely used for solving biomechanical optimal control
problems. For example, OpenSim now includes Moco~\cite{Dembia2020} which allows
users to find optimal control solutions with lower overhead. One advantage of
direct collocation is the speed at which solutions can be found, which can be
orders of magnitude faster than methods which require sequentially integrating
the differential equations. Direct collocation methods can still result in
overall slower speed when searching for global minima and choices in the
software design can reduce computational efficiency. For example, Moco relies on
finite differences for gradients which may reduce performance and accuracy.

Solving an optimal control problem resulting from direct collocation requires
evaluating the system's equations of motion, its Jacobian, and possibly its
Hessian thousands to millions of times, but this evaluation is parallelizable
over multiple compute cores and can leverage the massive sparsity of the
equations. Maximizing the compute and memory performance of the functions that
evaluate the constraints and their derivatives allow the subsequent
optimization algorithms to solve very large problems, both in simulation
duration and number of degrees of freedom.

Early dynamics simulation software used computer-aided algebra techniques, which
allowed for analytical derivatives when forming equations of motion, but this
has faded from popularity with the growth of numerical physics engines. Code
generated from computer-aided algebra can be optimized for evaluation
performance using both compiler pre-optimizations and built-in compiler
optimizations that can be harder to apply to numeric physics engines.

We will demonstrate software that leverages symbolic formulations to generate
computationally efficient parallelized functions based on implicit dynamics to
evaluate the non-linear programming problem's constraints and its Jacobian based
on the ideas of \cite{vandenBogert2011a}. We demonstrate solving a 3D arm-muscle
driven bicycle-rider model that has three degrees of freedom and seven algebraic
constraints resulting in millions of floating point operations in the implicit
equations of motion.

\begin{table*}[t]
  \centering
  \caption{\small Mean performance of seven bicycle-rider \(N=200\) optimal
  control problem (OCP) solutions on a Macbook Pro with 2.4 GHz 8-Core
  processor, running Python 3.12.9, SymPy 1.13.3, and opty 1.4.0 using Ipopt
  3.14.17 with Mumps 5.7.3.}
  \scriptsize
  \begin{tabular}{lllllllllll}
    \toprule
    Symbolic &
    Symbolic &
    Constraint &
    Jacobian &
    Symbolic &
    OCP &
    NLP &
    Objective &
    Gradient &
    Constraint &
    Jacobian
    \\
    EoM [s] &
    Jacobian [s] &
    Clang [s] &
    Clang [s] &
    OCP [s] &
    Solve [s] &
    iterations &
    evaluations &
    evaluations &
    evaluations &
    evaluations
    \\
    \midrule
    14.9 &
    35.4 &
    58.3 &
    66.7 &
    169.8 &
    73.0 &
    286 &
    1098 &
    286 &
    1098 &
    292
    \\
    \bottomrule
  \end{tabular}
  \label{tab:performance}
\end{table*}

\vspace{-1.5em}
\section*{METHODS}
\vspace{-1em}
%
The non-linear equations of motion for musculoskeletal-vehicle systems can be
described by differential algebraic equations that are functions of the states
\(\mathbf{y} \in \mathbb{R}^n\), controls \(\mathbf{r}\), and constant
parameters \(\mathbf{p}\), in general. The implicit equations take this form:
%
\begin{align}
  \mathbf{f}(\dot{\mathbf{y}}(t), \mathbf{y}(t), \mathbf{r}(t), \mathbf{p}) =
  \mathbf{0} \in \mathbb{R}^n
\end{align}
%
These equations can contain elements such as musculotendon force-velocity
relationships, kinematic loops, friction and collision forces, etc. We write
these equations as analytically differentiable functions. To map these to
non-linear programming constraints we discretize the equations with backward
Euler differentiation over a constant time step and code generate functions in
C using opty~\cite{Moore2018} that evaluate the constraints and Jacobian over
all \(N\) nodes while exploiting the high sparsity of the Jacobian. The
symbolic Jacobian is calculated in forward mode applying the chain rule through
all common sub-expressions which efficiently handles differentiating equations
of motion with many mathematical operations. This results in cacheable
computationally efficient numerical functions that evaluate the non-linear
programming problem constraints and its Jacobian. Evaluation of the objective is
generally a low computational cost, so this is only done in Python, but also
with a symbolic-to-numeric conversion.

To demonstrate our methods, we developed a bicycle-rider model that is the
non-linear Carvallo-Whipple model of the vehicle with four additional rigid
bodies representing the upper and lower arm body segments using
SymBRiM~\cite{Stienstra2023a}. We include four lumped Hill-type muscle models
representing the extensor and flexor groups for the elbow. The model has seven
holonomic constraints and four nonholonomic constraints resulting in a
10\textsuperscript{th} order model with seven additional algebraic equations.
This results in 17 equations with 2.8M floating point operations in the implicit
symbolic equations of motion. After efficient Jacobian formulation and compiler
pre-optimizations, this reduces to 10K and 69K operations for evaluating the
constraints and Jacobian, respectively. This numerical evaluation time depends
on the number of nodes, so proportional to \(N\times\)79K. For \(N=\)~10K we
time the evaluation of these two functions with and without parallelization
using OpenMP 4.5 seeing a 4.6\(\times\) and 3.1\(\times\) speed up over 23 ms
and 190 ms, respectively. Lastly, we define a minimal muscle excitation lane
change tracking task, starting from zero speed up to a mean speed of 3 m/s over
the fixed duration of 6 seconds, as an optimal control problem.

% 10K nodes on Jason's laptop
% single thread:
% In [2]: %timeit problem.constraints(initial_guess)
% 64 ms ± 4.21 ms per loop (mean ± std. dev. of 7 runs, 10 loops each)
% In [3]: %timeit problem.jacobian(initial_guess)
% 500 ms ± 44.6 ms per loop (mean ± std. dev. of 7 runs, 1 loop each)
% parallel:
% In [3]: %timeit problem.constraints(initial_guess)
% 25 ms ± 3.36 ms per loop (mean ± std. dev. of 7 runs, 10 loops each)
% In [4]: %timeit problem.jacobian(initial_guess)
% 253 ms ± 38.8 ms per loop (mean ± std. dev. of 7 runs, 1 loop each)

% 10K nodes on Sam's laptop
% single thread:
% In [1]: %timeit data.problem.constraints(data.initial_guess)
% 23.0 ms ± 553 μs per loop (mean ± std. dev. of 7 runs, 10 loops each)
% In [1]: %timeit data.problem.jacobian(data.initial_guess)
% 189 ms ± 1.09 ms per loop (mean ± std. dev. of 7 runs, 10 loops each)
% parallel:
% In [4]: %timeit data.problem.constraints(data.initial_guess)
% 5.05 ms ± 212 μs per loop (mean ± std. dev. of 7 runs, 100 loops each)
% In [9]: %timeit data.problem.jacobian(data.initial_guess)
% 60.2 ms ± 1.4 ms per loop (mean ± std. dev. of 7 runs, 10 loops each)

\vspace{-1.5em}
\section*{RESULTS AND DISCUSSION}
\vspace{-1em}
\pagestyle{empty}
%
Table~\ref{tab:performance} shows timings for solving a 200 node lane change
problem. Building the model and forming the OCP takes 185 seconds with most time
spent compiling the generated functions with Clang. This is a fixed cost
regardless of the size of \(N\) and can be cached to disk and need not be built
again unless the dynamics model changes. Ipopt solves the problem in 73 seconds.
Fig.~\ref{fig:trajectories} shows an example optimal solution.
%
\begin{figure}
    \centering
    \includegraphics[width=\linewidth]{figures/arm-muscle-bicycle-excitation.png}
    \caption{\small An optimal solution: 1) bicycle motion, 2) left/right
    arm biceps/triceps muscle excitations, 3) bicycle speed (solid) and
    propulsion torque (dashed), and 4) lateral displacement.}
    \label{fig:trajectories}
\end{figure}

\vspace{-1.5em}
\section*{CONCLUSIONS}
\vspace{-1em}
%
Our methods focus on maximizing the performance of evaluating the constraint and
Jacobian numerical functions. The performance of these functions scales linearly
with the number of collocation nodes and can be sped up multiplicatively by the
number of compute cores available in the best-case scenario. When Ipopt needs to
evaluate these functions thousands or millions of times, for example in a
long-duration problem, the overall compute time can be greatly reduced.

\vspace{-1em}
\bibliographystyle{abbrv}
\bibliography{references}

\vspace{-1em}
\section*{ACKNOWLEDGMENTS}
\vspace{-1em}
%
P. Stahlecker contributed to opty; funds were from CZI.

\end{document}
