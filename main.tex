\documentclass[11pt,twocolumn]{article}

\usepackage[includehead,margin=0.5in]{geometry}
\usepackage{fancyhdr}
\usepackage{amsmath}
\usepackage{graphicx}
\usepackage{parskip}  % disable paragraph indent
\usepackage{helvet}
\renewcommand{\familydefault}{\sfdefault}

\makeatletter
\renewcommand{\fnum@figure}{Fig \thefigure}
\makeatother

\title{\textbf{
  FAST SYMBOLIC METHODS FOR MUSCLE-DRIVEN\\
  OPTIMAL CONTROL AND PARAMETER IDENTIFICATION
}}
\author{
Jason K. Moore\textsuperscript{1}\and
Samuel G. Brockie\textsuperscript{2}\and
Timótheüs J. Stienstra\textsuperscript{1}\and
Antonie van den Bogert\textsuperscript{3}\\
\textsuperscript{1}Department of BioMechanical Engineering, Delft University of Technology, The Netherlands\\
\textsuperscript{2}Department of Engineering, University of Cambridge, United Kingdom\\
\textsuperscript{3}Mechanical Engineering, Cleveland State University, USA\\
Email: j.k.moore@tudelft.nl}
\date{}

\renewcommand{\thispagestyle}[1]{} % do nothing
\renewcommand{\headrulewidth}{0pt}

\begin{document}
\pagestyle{fancy}
\lhead{}
\rhead{\textbf{XX International Symposium on Computer Simulation in Biomechanics\\
July 23rd – 25th 2025, Uppsala}}
\maketitle
\section*{INTRODUCTION}
%
Direct collocation is now widely used for solving biomechanical optimal control
problems. For example, OpenSim now includes Moco~\cite{Dembia2019} which allows
users to find optimal control solutions with lower overhead. One advantage of
direct collocation when compared to shooting optimization methods is the speed
at which solutions can be found. The computation costs can be orders of
magnitude faster with direct collocation. But direct collocation formulations
rely on local minimization techniques and computational cost may still be high
when searching for global minima. There are computational choices in the design
of direct collocation methods that can reduce computational efficiency also, for
example, Moco relies on finite differences for gradients which can reduce
optimization solution performance.

Solving an optimal control problem resulting from direct collocation requires
evaluating the system's equations of motion, its Jacobian, and possibly its
Hessian thousands to millions of times for one solution, but this evaluation is
easily parallelizable over multiple compute cores and can leverage the massive
sparsity of the equations. This cannot be, in general, done in shooting methods
that require sequential integration stepping. Maximizing the compute and memory
performance of the functions that evaluate the constraints and their derivatives
allows the subsequent optimization algorithms to solve very large problems, both
in simulation duration and number of degrees of freedom.

Early dynamics simulation software used computer-aided algebra techniques, which
allowed for analytical derivatives when forming equations of motion, but this
has faded from popularity with the growth of numerical physics engines. Code
generated from computer aided algebra can be optimized for evaluation
performance using compiler pre-optimizations and built-in compiler optimizations
that are harder, or even impossible, to apply to numeric physics engines.

We will demonstrate software that leverages symbolic formulations to generate
computationally efficient parallelized functions based on implicit dynamics to
evaluate the non-linear programming problem's constraints and its derivatives
based on the ideas of \cite{vandenBogert2011a}. We will show a 3D arm-muscle
driven bicycle-rider model that has seven algebraic constraints and 2.8 million
floating point operations in the equations of motion.
%
\begin{table*}[t]
  \centering
  \caption{Mean performance values for the arm-muscle driven bicycle model lane
  change maneuver on a Macbook Pro with 2.4 GHz 8-Core processor, running Python 3.12.9, SymPy 1.13.3, and opty 1.4.0.}
  \scriptsize
  \begin{tabular}{lllllllllll}
    Symbolic & Symbolic & OCP & Constraint & Jacobian  & Jacobian & NLP & Objective &  Gradient & Constraint & Jacobian \\
    EOM & OCP & Solve & Compilation &  Compilation & Differentiation & iterations & evaluations & evaluations & evaluations & evaluations \\
    14.9 & 169.8 & 73.0 & 58.3 & 66.7 & 35.4 & 286 & 1098 & 286 & 1098 & 292
  \end{tabular}
  \label{tab:performance}
\end{table*}

\section*{METHODS}
%
The non-linear equations of motion for musculoskeletal system can be described
by differential algebraic equations that are a function of the state \(n\)
states \(\mathbf{y}\), controls \(\mathbf{r}\), and constant parameters
\(\mathbf{p}\), in general. These equations take this form:
%
\begin{align}
  \mathbf{f}(\dot{\mathbf{y}}(t), \mathbf{y}(t), \mathbf{r}(t), \mathbf{p}) =
  \mathbf{0} \in \mathrm{R}^n
\end{align}
%
These equations can contain elements such as musculotendon force-velocity
relationships, kinematic loops, friction and collision forces, etc. We write
these equations as analytically differentiable functions. To map these to
non-linear programming constraints we discretize the equations with backward
Euler differentiation over a constant time step and code generate functions in C
that evaluate the constraints and Jacobian over all \(N\) nodes, while
exploiting the high sparsity of the Jacobian~\cite{Moore2018}. The symbolic
Jacobian is calculated in forward mode applying the chain rule through all
common sub-expressions which handles equations of motion with large or many
operations. This results in cacheable computationally efficient numerical
functions that evaluate the non-linear programming problem constraints and its
Jacobian. Evaluation of the objective is generally a low computational cost, so
this is only done in Python, also with a symbolic to numeric conversion.

\section*{RESULTS AND DISCUSSION}
%
To demonstrate the qualities of our methods, we have developed a bicycle-rider
model that includes the non-linear Carvallo-Whipple of the vehicle with four
additional rigid bodies representing the upper and lower arm body
segments~\cite{Stienstra2023a}. The arms are added in such a way that no
additional degrees of freedom are added to the vehicle model due to introduced
closed-loop kinematic constraints. The model has seven holonomic constraints and
four nonholonomic constraints resulting in a 10\textsuperscript{th} order model.
We include four lumped Hill-Type muscle models representing the extensor and
flexor groups for the elbow. This results in a set of differential algebraic
equations with 2.8M floating point operations in the implicit symbolic equations
of motion. After efficient Jacobian formulation and compiler pre-optimizations,
this reduces to 10K and 69K operations, for the constraints and Jacobian
respectively. Once the constraints and Jacobian NLP numerical evaluation
functions are rewritten the evaluation time depends on the number of nodes, so
\(N\times\)79K. For X nodes we time the evaluation of these two functions with
and without parallelization using from OpenMP. We then solve the optimal control
problem defined by the objective to follow a lane change path with the front
wheel.

\begin{figure}
    \centering
    \includegraphics[width=\linewidth]{figures/arm-muscle-bicycle-excitation.png}
    \caption{Triceps and biceps muscle group excitations for the left and right arms to execute a lane change maneuver.}
    \label{fig:enter-label}
\end{figure}

The timings are shown in Table~\ref{tab:performance}. Building the model and
forming the optimal control problem takes 185 seconds with most time spent
compiling the generated functions with Clang. This can be cached to disk and
need not be built again unless the dynamics model changes. IPOPT solves the
problem to an acceptable level in 73 seconds.

\section*{CONCLUSIONS}
%
Our methods maximize the performance of evaluating the optimal control
constraint and Jacobian numerical functions. The performance of these functions
scale linearly with the number of collocation nodes and can be sped up
approximately multiplicatively by the number of compute cores available.
Whenever IPOPT needs to evaluate these functions thousands or millions of times
the overall solution can be minimized.

\bibliography{references}
\bibliographystyle{plain}

\section*{ACKNOWLEDGEMENTS}
Peter Stahlecker

\end{document}