\documentclass[11pt,twocolumn]{article}

\usepackage[includehead,margin=0.5in]{geometry}
\usepackage{fancyhdr}
\usepackage{amsmath}
\usepackage{helvet}
\renewcommand{\familydefault}{\sfdefault}

\title{\textbf{
  FAST SYMBOLIC METHODS FOR MUSCLE-DRIVEN\\
  OPTIMAL CONTROL AND PARAMETER IDENTIFICATION
}}
\author{
Jason K. Moore \and
Samuel G. Brockie \and
Timótheüs J. Stienstra \and
Antonie van den Bogert \\
Department of BioMechanical Engineering, Delft University of Technology, The Netherlands\\
Email: j.k.moore@tudelft.nl}
\date{}

\renewcommand{\thispagestyle}[1]{} % do nothing

\begin{document}
\pagestyle{fancy}
\lhead{}
\rhead{XX International Symposium on Computer Simulation in Biomechanics\\
July 23rd – 25th 2025, Uppsala} %empty
\maketitle
\section*{INTRODUCTION}
%
Direct collocation has grown rapidly for solving biomechanical optimal control
problems. For example, OpenSim now includes Moco~\cite{Dembia2019} which allows
users to find solutions with relatively low user overhead. One advantage of
direct collocation when compared to shooting optimization methods is the speed
at which solutions can be found. The computation costs can be orders of
magnitude faster with direct collocation. But direct collocation methods rely on
local minimization techniques and computational cost can still increase when
searching for global minima.

Direct collocation methods require evaluating the system's equations of motion,
its Jacobian, and possibly its Hessian thousands to millions of times for one
solution, but this evaluation is easily parallelizable over multiple compute
cores and can leverage the massive sparsity of the equations. This cannot be, in
general, done in shooting methods that require sequential integration stepping.
Maximizing the compute and memory performance of the functions that evaluate the
constraints and their derivatives allows the subsequent optimization algorithms
to solve very large problems, both in simulation duration and number of degrees
of freedom.

Early dynamics simulation software used computer-aided algebra techniques, which
allowed for analytical derivatives when forming equations of motion, but this
has faded from popularity with growth of numerical physics engines. Code
generated from computer aided algebra can be optimized for evaluation
performance using compiler pre-optimizations and built-in compiler optimizations
that are harder, or impossible, to apply to numeric physics engines.

We will demonstrate software that leverages symbolic formulations to generate
computationally efficient parallelized functions to evaluate the non-linear
programming problem's constraints and its derivatives. We will show an
arm-muscle driven bicycle-rider model that has XXX million floating point
operations for evaluating the constraints and its Jacobian.

\section*{METHODS}
%
The non-linear equations of motion for musculoskeletal system can be described
by differential algebraic equations, in general. These equations take this
form:
%
\begin{align}
  \mathbf{f}(\dot{\mathbf{y}}(t), \mathbf{y}(t), \mathbf{r}(t), \mathbf{p}) =
  \mathbf{0} \in \mathrm{R}^n
\end{align}
%
These equations can contain elements such as musculotendon force-velocity
relationships, kinematic loops, friction, collision, etc. We write these
equations as analytically differentiable functions. The Jacobian and Hessian can
be formed by recursive symbolic differentiation. To map these to non-linear
programming constraints we discretize the equations with backward Euler
differentiation over a constant time step and code generate functions in C that
evaluate the constraints and Jacobian over all nodes, while exploiting the high
sparsity of the Jacobian~\cite{Moore2018}. The symbolic Jacobian is calculated
in forward mode applying the chain rule through all common sub-expressions which
handles large equations of motion. This results in cacheable computationally
efficient numerical functions that evaluate the non-linear programming problem
constraints and its Jacobian. Evaluation of the objective is generally a low
computational cost, so this is only done in Python, also with a symbolic to
numeric conversion.
%
\begin{table*}[t]
  \centering
  \caption{Caption}
  \tiny
  \begin{tabular}{lllllllllll}
    Symbolic & Symbolic & OCP & Constraint & Jacobian  & Jacobian & NLP & Objective &  Gradient & Constraint & Jacobian \\
    EOM & OCP & Solve & Compilation &  Compilation & Differentiation & iterations & evaluations & evaluations & evaluations & evaluations \\
    14.9 & 169.8 & 73.0 & 58.3 & 66.7 & 35.4 & 286 & 1098 & 286 & 1098 & 292
  \end{tabular}
  \label{tab:my_label}
\end{table*}

\section*{RESULTS AND DISCUSSION}
%
To demonstrate the qualities of our methods we have developed a bicycle-rider
model that includes the non-linear Carvallo-Whipple of the vehicle and with four
additional rigid bodies representing the upper and lower arm body
segments~\cite{Stienstra2023a}. The arms are added in such a way that no
additional degrees of freedom are added to the vehicle model due to the
closed-loop kinematic constraints. We include four lumped Hill-Type muscle
models reprsenting the extensor and flexor groups for the elbow. This results in
a set of differential algebraic equations with 2.8M floating point operations in
the symbolic equations of motion. The Jacobian of the equations of motion with
respect to the states an muscle excitation inputs has XXM operations. After
compiler preoptimizations, this reduces to XK and XM, respectively. Once the
constraints and Jacobian NLP numerical evaluation functions are rewritten the
evaluation time depends on the number of nodes. For X nodes we time the
evaluation of these two functions with and without parallelization using from
OpenMP. The timings are shown in Table~\ref{TODO}.

We then solve the optimal control problem defined by the objective to follow a
lane change path with the front wheel.
test
\section*{CONCLUSIONS}

\bibliography{references}
\bibliographystyle{plain}

\end{document}
