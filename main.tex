\documentclass[11pt,twocolumn]{article}

\usepackage[margin=0.5in]{geometry}
\usepackage{helvet}
\renewcommand{\familydefault}{\sfdefault}

\title{Fast Symbolic Methods for Muscle-Driven\\Optimal Control and Parameter
Identification}
\author{Jason K. Moore}
\date{\today}

\begin{document}
\maketitle
\section*{INTRODUCTION}
%
Direct collocation has grown rapidly for solving biomechanical optimal control
problems. For example, Opensim now includes Moco which allows users to find
solutions with relatively low user overhead. An advantage of direct collocation
compared to shooting is the speed at which solutions can be found.  The
computation costs can be an orders of magnitude faster with direct collocation.
But direct collocation methods are ultimately local minimzers and computational
cost can increase when searching for global minima.

Direct collocation methods require evaluating the system's equations of motion,
its Jacobian, and possibly its Hessian thousands to millions of times for one
solution, but this evaluation is eaisly parallizable over mutliple compute
cores and can leverage the massive sparsity of the equations. This cannot be,
in general, done in shooting methods that require sequential integration
stepping. Maximizing the compute and memory performance of the functions that
evaluate the constraints and their derivatives allows the subseqent optimization
algorithms to solve very large problems, both in simulation duration and number
of degrees of freedom.

Early, dynamics simulation software used computer aided algebra techniques,
which allowed for analytical derivatives when forming equations of motion, but
this has faded from popularity with growth of numerical physics engines. Code
generated from computer aided algebra can be optimized for evaluation
performance using compiler pre-optimizations and built-in compiler
optimizations that are unlikley available to apply to numeric physics engines.

We will demonstrate software that leverages symbolic formulations to generate
computationally efficient parallelized functions to evaluate the non-linear
programming problem's constraints and its derivatives. We will show an
arm-muscle driven bicycle-rider model that has XM operations for evaluating the
constraints and its Jacobian.

\section*{METHODS}
Explain the formulation of the NLP problem from a symbolic optimal control
description.

\section*{RESULTS AND DISCUSSION}

\section*{CONCLUSIONS}

\end{document}
